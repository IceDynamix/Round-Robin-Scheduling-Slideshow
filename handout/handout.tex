\documentclass{article}
\usepackage[utf8]{inputenc}
\usepackage[T1]{fontenc}

\usepackage[ngerman]{babel}
\usepackage{adjustbox}
\usepackage{appendixnumberbeamer}

\usepackage{xcolor}
\definecolor{red}{RGB}{235,42,27}
\definecolor{yellow}{RGB}{187,173,10}
\definecolor{purple}{RGB}{89,30,159}
\definecolor{blue}{RGB}{20,125,145}

\usepackage[backend=biber,sorting=none]{biblatex}
\addbibresource{../bib/bib.bib}

% THEME ------------------------------------

% https://github.com/matze/mtheme
% Note: Compile with XeLaTeX for the cool font
\usetheme{metropolis}


% Metadata ------------------------------------

\title{Scheduling 2}
\subtitle{Round Robin}
\author{******** Nguyen}
\date{9. November 2020}


\begin{document}

\newcommand{\PZero}{\textcolor{red}{P0}}
\newcommand{\POne}{\textcolor{yellow}{P1}}
\newcommand{\PTwo}{\textcolor{purple}{P2}}
\newcommand{\PThree}{\textcolor{blue}{P3}}

\maketitle

\section{Konzept}

Es wird ein Quantum (eine Art Zeitslot) festgelegt, welcher üblicherweise zwischen 10ms und 100ms liegt. Wenn ein Prozess länger braucht als ein Quantum, wird es wieder hinten an die Warteschlange geschickt. Wenn ein Prozess während eines Quantums fertig wird, bekommt der nächste ein vollständiges Quantum.

\section{Beispiel}

Es sei ein Quantum von 100ms festgelegt.

\begin{table}[!h]
    \centering
    \begin{tabular}{c|c|c}
        \textbf{Prozess} & \textbf{Ankunftszeit} & \textbf{Prozessdauer} \\
        \hline{}
        \PZero{}         & 0ms                   & 150ms                 \\
        \POne{}          & 30ms                  & 250ms                 \\
        \PTwo{}          & 120ms                 & 50ms                  \\
        \PThree{}        & 130ms                 & 170ms
    \end{tabular}
    \caption{Beispiel Prozesse}
\end{table}

\begin{table}[!h]
    \centering
    \begin{tabular}{c|cccc|c|cccc}
        \multicolumn{6}{c}{\textbf{Zeitstrahl}} & \multicolumn{4}{|c}{\textbf{Verbleibende Zeit}}                                                                                                                                                         \\ \hline
        \textbf{Zeit}                           & \textbf{Slot 1}                                 & \textbf{Slot 2} & \textbf{Slot 3} & \textbf{Slot 4} & \textbf{Event}   & \textbf{\PZero{}} & \textbf{\POne{}} & \textbf{\PTwo{}} & \textbf{\PThree{}} \\ \hline
        0ms                                     & \PZero{}                                        &                 &                 &                 & \PZero{} join    & \textbf{150}      &                  &                  &                    \\
        30ms                                    & \PZero{}                                        & \POne{}         &                 &                 & \POne{} join     & \textbf{120}      & 250              &                  &                    \\
        100ms                                   & \POne{}                                         & \PZero{}        &                 &                 & \textbf{Rotate}  & \textbf{50}       & 250              &                  &                    \\
        120ms                                   & \POne{}                                         & \PZero{}        & \PTwo{}         &                 & \PTwo{} join     &                   & \textbf{230}     & 50               &                    \\
        130ms                                   & \POne{}                                         & \PZero{}        & \PTwo{}         & \PThree{}       & \PThree{} join   &                   & \textbf{220}     &                  & 170                \\
        200ms                                   & \PZero{}                                        & \PTwo{}         & \PThree{}       & \POne{}         & \textbf{Rotate}  & 50                & \textbf{150}     &                  &                    \\
        250ms                                   & \PTwo{}                                         & \PThree{}       & \POne{}         &                 & \PZero{} finish  & \textbf{0}        &                  & 50               &                    \\
        % P2 würde hier ein vollständiges Quantum bekommen anstatt due 50ms die vom vorherigen übrig bleiben
        300ms                                   & \PThree{}                                       & \POne{}         &                 &                 & \PTwo{} finish   &                   &                  & \textbf{0}       & 170                \\
        400ms                                   & \POne{}                                         & \PThree{}       &                 &                 & \textbf{Rotate}  &                   & 150              &                  & \textbf{70}        \\
        500ms                                   & \PThree{}                                       & \POne{}         &                 &                 & \textbf{Rotate}  &                   & \textbf{50}      &                  & 70                 \\
        570ms                                   & \POne{}                                         &                 &                 &                 & \PThree{} finish &                   & 50               &                  & \textbf{0}         \\
        620ms                                   &                                                 &                 &                 &                 & \POne{} finish   &                   & \textbf{0}       &                  &
    \end{tabular}
    \caption{Ablauf im Round Robin Scheduling}
\end{table}

\section{Aufgabe}

\subsection{Aufgabenstellung}

\begin{table}[!h]
    \begin{tabular}{c|c|c}
        \textbf{Prozess} & \textbf{Ankunftszeit} & \textbf{Prozessdauer} \\
        \hline{}
        \PZero{}         & 4ms                   & 317ms                 \\
        \POne{}          & 105ms                 & 16ms                  \\
        \PTwo{}          & 220ms                 & 186ms
    \end{tabular}
    \caption{Gegebene Prozesse}
\end{table}

\begin{enumerate}
    \item Bestimme die Wartezeit (Wie viel länger der Prozess als die Prozessdauer braucht) für jeden Prozess mit einem Quantum von 100ms
    \item Vergleiche die Wartezeiten mit der eines FCFS Algorithmus
\end{enumerate}

\subsection{Lösung}

\begin{table}[!h]
    \centering
    \begin{tabular}{c|cc|c|ccc}
        \multicolumn{4}{c}{\textbf{Zeitstrahl}} & \multicolumn{3}{|c}{\textbf{Verbleibende Zeit}}                                                                                                                                  \\ \hline
        \textbf{Zeit}                           & \textbf{Slot 1}                                 & \textbf{Slot 2} & \textbf{Event}                 & \multicolumn{1}{l}{\textbf{\PZero{}}} & \textbf{\POne{}} & \textbf{\PTwo{}} \\ \hline
        4ms                                     & \PZero{}                                        &                 & \PZero{} join                  & \textbf{317}                          &                  &                  \\
        104ms                                   & \PZero{}                                        &                 & \textbf{Rotate}                & \textbf{217}                          &                  &                  \\
        % P0 bekommt hier zwei Quantums da niemand bislang in der Queue war
        105ms                                   & \PZero{}                                        & \POne{}         & \POne{} join                   & \textbf{216}                          & 16               &                  \\
        204ms                                   & \POne{}                                         & \PZero{}        & \textbf{Rotate}                & \textbf{117}                          &                  &                  \\
        220ms                                   & \PZero{}                                        & \PTwo{}         & \POne{} finish \& \PTwo{} join &                                       & \textbf{0}       & 186              \\
        % P1 endet gleichzeitig zum Join von P2, P1 wird in diesem Beispiel abgearbeitet bevor \PTwo{} hinzugefuegt wird
        330ms                                   & \PTwo{}                                         & \PZero{}        & \textbf{Rotate}                & \textbf{17}                           &                  &                  \\
        430ms                                   & \PZero{}                                        & \PTwo{}         & \textbf{Rotate}                &                                       &                  & \textbf{86}      \\
        447ms                                   & \PTwo{}                                         &                 & \PZero{} finish                & \textbf{0}                            &                  &                  \\
        533ms                                   &                                                 &                 & \PTwo{} finish                 &                                       &                  & \textbf{0}
    \end{tabular}
    \caption{Ablauf}
\end{table}

\begin{table}[!h]
    \centering
    \begin{tabular}{c|c|c|c|c}
        \textbf{Prozess} & \textbf{Start} & \textbf{Ende} & \textbf{Wartezeit (RR)} & \textbf{Wartezeit (FCFS)} \\
        \hline
        \PZero{}         & 4ms            & 447ms         & 126ms                   & 0ms                       \\
        \POne{}          & 105ms          & 220ms         & 99ms                    & 342ms                     \\
        \PTwo{}          & 220ms          & 533ms         & 127ms                   & 0ms
    \end{tabular}
    \caption{Wartezeit und Vergleich mit FCFS}
\end{table}

\pagebreak

\section{Fazit}

\begin{itemize}
    \item Alle Prozesse werden hier mit der gleichen Dringlichkeit bearbeitet (kann aber auch keine Prioritäten setzen)
    \item Es wird sicher gegangen, dass kein Prozess verhungert
    \item Niedrige durchschnittliche Verweilzeit für alle Prozesse
    \item Algorithmus ist leicht zu implementieren
    \item Wichtig: Das Quantum muss gut gewählt sein
          \begin{itemize}
              \item Wenn das Quantum zu groß ist, ähnelt es einem First Come First Serve (FCFS/FIFO)
              \item Wenn das Quantum zu klein ist, ähnelt es einem Shortest Job First (SJF) und der Kontextwechsel Aufwand wird größer
          \end{itemize}
\end{itemize}

\nocite{*}
\printbibliography

\end{document}
