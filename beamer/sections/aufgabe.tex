\section{Aufgabe}

\begin{frame}
    \frametitle{Zeitberechnung}

    \begin{table}[]
        \begin{tabular}{c|c|c}
            \textbf{Prozess} & \textbf{Ankunftszeit} & \textbf{Prozessdauer} \\
            \hline{}
            \PZero{}         & 4ms                   & 317ms                 \\
            \POne{}          & 105ms                 & 16ms                  \\
            \PTwo{}          & 220ms                 & 186ms
        \end{tabular}
    \end{table}

    \begin{block}{Aufgabe}
        \begin{enumerate}
            \item Bestimme die Wartezeit (Wie viel l{\"a}nger der Prozess als die Prozessdauer braucht) f{\"u}r jeden Prozess mit einem Quantum von 100ms
            \item Vergleiche die Wartezeiten mit der eines FCFS Algorithmus
        \end{enumerate}
    \end{block}

\end{frame}

\begin{frame}
    \frametitle{Wartezeiten}

    \begin{table}[]
        \begin{adjustbox}{width=0.8\textwidth}
            \begin{tabular}{c|cc|c|ccc}
                \multicolumn{4}{c}{\textbf{Zeitstrahl}} & \multicolumn{3}{|c}{\textbf{Verbleibende Zeit}}                                                                                                                                  \\ \hline
                \textbf{Zeit}                           & \textbf{Slot 1}                                 & \textbf{Slot 2} & \textbf{Event}                 & \multicolumn{1}{l}{\textbf{\PZero{}}} & \textbf{\POne{}} & \textbf{\PTwo{}} \\ \hline
                4ms                                     & \PZero{}                                        &                 & \PZero{} join                  & \alert{317}                           &                  &                  \\
                104ms                                   & \PZero{}                                        &                 & \alert{Rotate}                 & \alert{217}                           &                  &                  \\
                % P0 bekommt hier zwei Quantums da niemand bislang in der Queue war
                105ms                                   & \PZero{}                                        & \POne{}         & \POne{} join                   & \alert{216}                           & 16               &                  \\
                204ms                                   & \POne{}                                         & \PZero{}        & \alert{Rotate}                 & \alert{117}                           &                  &                  \\
                220ms                                   & \PZero{}                                        & \PTwo{}         & \POne{} finish \& \PTwo{} join &                                       & \alert{0}        & 186              \\
                % P1 endet gleichzeitig zum Join von P2, P1 wird in diesem Beispiel abgearbeitet bevor \PTwo{} hinzugefuegt wird
                330ms                                   & \PTwo{}                                         & \PZero{}        & \alert{Rotate}                 & \alert{17}                            &                  &                  \\
                430ms                                   & \PZero{}                                        & \PTwo{}         & \alert{Rotate}                 &                                       &                  & \alert{86}       \\
                447ms                                   & \PTwo{}                                         &                 & \PZero{} finish                & \alert{0}                             &                  &                  \\
                533ms                                   &                                                 &                 & \PTwo{} finish                 &                                       &                  & \alert{0}
            \end{tabular}
        \end{adjustbox}
    \end{table}

    \pause

    \begin{table}[]
        \begin{tabular}{c|c|c|c|c}
            \textbf{Prozess} & \textbf{Start} & \textbf{Ende} & \textbf{Wartezeit (RR)} & \textbf{Wartezeit (FCFS)} \\
            \hline
            \PZero{}         & 4ms            & 447ms         & 126ms                   & 0ms                       \\
            \POne{}          & 105ms          & 220ms         & 99ms                    & 342ms                     \\
            \PTwo{}          & 220ms          & 533ms         & 127ms                   & 0ms
        \end{tabular}
    \end{table}

\end{frame}
