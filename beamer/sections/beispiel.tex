\section{Round Robin Beispiel}

\begin{frame}
    \frametitle{Beispiel mit 4 Prozessen}

    % Angenommen Quantum = 100ms

    \begin{table}[]
        \begin{adjustbox}{width=0.4\textwidth}
            \begin{tabular}{c|c|c}
                \textbf{Prozess} & \textbf{Ankunftszeit} & \textbf{Prozessdauer} \\
                \hline{}
                P0               & 0ms                   & 150ms                 \\
                P1               & 30ms                  & 250ms                 \\
                P2               & 120ms                 & 50ms                  \\
                P3               & 130ms                 & 170ms
            \end{tabular}
        \end{adjustbox}
    \end{table}

    \begin{table}[]
        \begin{adjustbox}{width=0.8\textwidth}
            \begin{tabular}{c|cccc|c|cccc}
                \multicolumn{6}{c}{\textbf{Zeitstrahl}} & \multicolumn{4}{|c}{\textbf{Verbleibende Zeit}}                                                                                                                                \\ \hline
                \textbf{Zeit}                           & \textbf{Slot 1}                                 & \textbf{Slot 2} & \textbf{Slot 3} & \textbf{Slot 4} & \textbf{Event} & \textbf{P0} & \textbf{P1} & \textbf{P2} & \textbf{P3} \\ \hline
                0ms                                     & \alert{P0}                                      &                 &                 &                 & P0 join        & \alert{150} &             &             &             \\
                \pause 30ms                             & P0                                              & \alert{P1}      &                 &                 & P1 join        & \alert{120} & 250         &             &             \\
                \pause 100ms                            & \alert{P1}                                      & \alert{P0}      &                 &                 & \alert{Rotate} & \alert{50}  & 250         &             &             \\
                \pause 120ms                            & P1                                              & P0              & \alert{P2}      &                 & P2 join        &             & \alert{230} & 50          &             \\
                \pause 130ms                            & P1                                              & P0              & P2              & \alert{P3}      & P3 join        &             & \alert{220} &             & 170         \\
                \pause 200ms                            & \alert{P0}                                      & \alert{P2}      & \alert{P3}      & \alert{P1}      & \alert{Rotate} & 50          & \alert{150} &             &             \\
                \pause 250ms                            & P2                                              & P3              & P1              &                 & P0 finish      & \alert{0}   &             & 50          &             \\
                % P2 würde hier ein vollständiges Quantum bekommen anstatt due 50ms die vom vorherigen übrig bleiben
                \pause 300ms                            & P3                                              & P1              &                 &                 & P2 finish      &             &             & \alert{0}   & 170         \\
                \pause 400ms                            & \alert{P1}                                      & \alert{P3}      &                 &                 & \alert{Rotate} &             & 150         &             & \alert{70}  \\
                \pause 500ms                            & \alert{P3}                                      & \alert{P1}      &                 &                 & \alert{Rotate} &             & \alert{50}  &             & 70          \\
                \pause 570ms                            & P1                                              &                 &                 &                 & P3 finish      &             & 50          &             & \alert{0}   \\
                \pause 620ms                            &                                                 &                 &                 &                 & P1 finish      &             & \alert{0}   &             &
            \end{tabular}
        \end{adjustbox}
    \end{table}

\end{frame}
