\section{Beispiel}

\begin{frame}
    \frametitle{Beispiel mit 4 Prozessen}

    % Angenommen Quantum = 100ms

    \begin{table}[]
        \begin{adjustbox}{width=0.4\textwidth}
            \begin{tabular}{c|c|c}
                \textbf{Prozess} & \textbf{Ankunftszeit} & \textbf{Prozessdauer} \\
                \hline{}
                \PZero{}         & 0ms                   & 150ms                 \\
                \POne{}          & 30ms                  & 250ms                 \\
                \PTwo{}          & 120ms                 & 50ms                  \\
                \PThree{}        & 130ms                 & 170ms
            \end{tabular}
        \end{adjustbox}
    \end{table}

    \begin{table}[]
        \begin{adjustbox}{width=0.8\textwidth}
            \begin{tabular}{c|cccc|c|cccc}
                \multicolumn{6}{c}{\textbf{Zeitstrahl}} & \multicolumn{4}{|c}{\textbf{Verbleibende Zeit}}                                                                                                                                                         \\ \hline
                \textbf{Zeit}                           & \textbf{Slot 1}                                 & \textbf{Slot 2} & \textbf{Slot 3} & \textbf{Slot 4} & \textbf{Event}   & \textbf{\PZero{}} & \textbf{\POne{}} & \textbf{\PTwo{}} & \textbf{\PThree{}} \\ \hline
                0ms                                     & \PZero{}                                        &                 &                 &                 & \PZero{} join    & \alert{150}       &                  &                  &                    \\
                \pause 30ms                             & \PZero{}                                        & \POne{}         &                 &                 & \POne{} join     & \alert{120}       & 250              &                  &                    \\
                \pause 100ms                            & \POne{}                                         & \PZero{}        &                 &                 & \alert{Rotate}   & \alert{50}        & 250              &                  &                    \\
                \pause 120ms                            & \POne{}                                         & \PZero{}        & \PTwo{}         &                 & \PTwo{} join     &                   & \alert{230}      & 50               &                    \\
                \pause 130ms                            & \POne{}                                         & \PZero{}        & \PTwo{}         & \PThree{}       & \PThree{} join   &                   & \alert{220}      &                  & 170                \\
                \pause 200ms                            & \PZero{}                                        & \PTwo{}         & \PThree{}       & \POne{}         & \alert{Rotate}   & 50                & \alert{150}      &                  &                    \\
                \pause 250ms                            & \PTwo{}                                         & \PThree{}       & \POne{}         &                 & \PZero{} finish  & \alert{0}         &                  & 50               &                    \\
                % P2 würde hier ein vollständiges Quantum bekommen anstatt due 50ms die vom vorherigen übrig bleiben
                \pause 300ms                            & \PThree{}                                       & \POne{}         &                 &                 & \PTwo{} finish   &                   &                  & \alert{0}        & 170                \\
                \pause 400ms                            & \POne{}                                         & \PThree{}       &                 &                 & \alert{Rotate}   &                   & 150              &                  & \alert{70}         \\
                \pause 500ms                            & \PThree{}                                       & \POne{}         &                 &                 & \alert{Rotate}   &                   & \alert{50}       &                  & 70                 \\
                \pause 570ms                            & \POne{}                                         &                 &                 &                 & \PThree{} finish &                   & 50               &                  & \alert{0}          \\
                \pause 620ms                            &                                                 &                 &                 &                 & \POne{} finish   &                   & \alert{0}        &                  &
            \end{tabular}
        \end{adjustbox}
    \end{table}

\end{frame}
